% Options for packages loaded elsewhere
\PassOptionsToPackage{unicode}{hyperref}
\PassOptionsToPackage{hyphens}{url}
%
\documentclass[
]{article}
\usepackage{lmodern}
\usepackage{amssymb,amsmath}
\usepackage{ifxetex,ifluatex}
\ifnum 0\ifxetex 1\fi\ifluatex 1\fi=0 % if pdftex
  \usepackage[T1]{fontenc}
  \usepackage[utf8]{inputenc}
  \usepackage{textcomp} % provide euro and other symbols
\else % if luatex or xetex
  \usepackage{unicode-math}
  \defaultfontfeatures{Scale=MatchLowercase}
  \defaultfontfeatures[\rmfamily]{Ligatures=TeX,Scale=1}
\fi
% Use upquote if available, for straight quotes in verbatim environments
\IfFileExists{upquote.sty}{\usepackage{upquote}}{}
\IfFileExists{microtype.sty}{% use microtype if available
  \usepackage[]{microtype}
  \UseMicrotypeSet[protrusion]{basicmath} % disable protrusion for tt fonts
}{}
\makeatletter
\@ifundefined{KOMAClassName}{% if non-KOMA class
  \IfFileExists{parskip.sty}{%
    \usepackage{parskip}
  }{% else
    \setlength{\parindent}{0pt}
    \setlength{\parskip}{6pt plus 2pt minus 1pt}}
}{% if KOMA class
  \KOMAoptions{parskip=half}}
\makeatother
\usepackage{xcolor}
\IfFileExists{xurl.sty}{\usepackage{xurl}}{} % add URL line breaks if available
\IfFileExists{bookmark.sty}{\usepackage{bookmark}}{\usepackage{hyperref}}
\hypersetup{
  pdftitle={Ärztliche Praxen werden zur Unterstützung der Impfzentren gebraucht},
  pdfauthor={Thomas Czihal, Lars Kroll, Edgar Steiger},
  hidelinks,
  pdfcreator={LaTeX via pandoc}}
\urlstyle{same} % disable monospaced font for URLs
\usepackage[margin=1in]{geometry}
\usepackage{graphicx,grffile}
\makeatletter
\def\maxwidth{\ifdim\Gin@nat@width>\linewidth\linewidth\else\Gin@nat@width\fi}
\def\maxheight{\ifdim\Gin@nat@height>\textheight\textheight\else\Gin@nat@height\fi}
\makeatother
% Scale images if necessary, so that they will not overflow the page
% margins by default, and it is still possible to overwrite the defaults
% using explicit options in \includegraphics[width, height, ...]{}
\setkeys{Gin}{width=\maxwidth,height=\maxheight,keepaspectratio}
% Set default figure placement to htbp
\makeatletter
\def\fps@figure{htbp}
\makeatother
\setlength{\emergencystretch}{3em} % prevent overfull lines
\providecommand{\tightlist}{%
  \setlength{\itemsep}{0pt}\setlength{\parskip}{0pt}}
\setcounter{secnumdepth}{-\maxdimen} % remove section numbering
\usepackage{booktabs}
\usepackage{longtable}
\usepackage{array}
\usepackage{multirow}
\usepackage{wrapfig}
\usepackage{float}
\usepackage{colortbl}
\usepackage{pdflscape}
\usepackage{tabu}
\usepackage{threeparttable}
\usepackage{threeparttablex}
\usepackage[normalem]{ulem}
\usepackage{makecell}
\usepackage{xcolor}

\title{Ärztliche Praxen werden zur Unterstützung der Impfzentren gebraucht}
\author{Thomas Czihal, Lars Kroll, Edgar Steiger}
\date{4 2 2021}

\begin{document}
\maketitle

\hypertarget{zusammenfassung}{%
\section{Zusammenfassung}\label{zusammenfassung}}

Mit den bevorstehenden Lieferungen der Impfstoffe wird es bald nötig,
zusätzlich zu den Impfzentren auch Impfungen in vertragsärztlichen
Praxen durchzuführen, um die vorhandenen Impfstoffe schnellstmöglich zu
verabreichen.

\hypertarget{annahmen-parameter-und-szenarien}{%
\section{Annahmen, Parameter und
Szenarien}\label{annahmen-parameter-und-szenarien}}

Wir modellieren mit 400 Impfzentren in Deutschland, die insgesamt
täglich 200.000 Impfungen durchführen können (Montag bis Sonntag).
Weiterhin rechnen wir mit etwa 50.000 vertragsärztlichen Praxen, in
denen mindestens jeweils 20 Impfungen pro Tag möglich sind (Montag bis
Freitag). Damit können in diesen Praxen theoretisch mehr als 700.000
Impfungen pro Tag im Wochenmittel bzw. 5 Millionen Impfungen pro Woche
durchgeführt werden. In unserer Modellierung gehen wir, davon aus, dass
die Praxen einspringen, um Lücken in der Kapazität der Impfzentren zu
füllen.

Für die zukünftige Planung berücksichtigen wir die verabredeten
Impfstofflieferungen je nach Hersteller, sowie die nach heutigem Stand
bereits verabreichten (RKI) und gelieferten Impfstoffdosen. In
erweiterten Leistungsfähigkeits-Szenarien berechnen wir unsere
Ergebnisse für Impfzentren mit erhöhter Leistungsfähigkeit (300.000
Impfungen gesamt pro Tag). Damit ergeben sich für die Impfzentren werk-
und wochentägliche Kapazitäten je nach Einrichtung wie in Abbildung 1
dargestellt.

\textbf{Abb. 1: Leistungsfähigkeit der Impfzentren}

\includegraphics{impfmodellierung_report_files/figure-latex/szenarientabelle-1.pdf}

Für die Lieferungen berücksichtigen wir zwei Lieferungs-Szenarien: Im
ersten Szenario werden die quartalsweise zugesicherten Lieferungen
gleichmäßig auf die Monate des Quartals verteilt (``Gleichverteilung''),
im zweiten Szenario (``Linearer Anstieg'') verschieben sich die
zugesagten Lieferungen bis Juni nach hinten, sodass im 2. Quartal im
April 20\%, im Mai 35\% und dafür im Juni 45\% der für das Quartal
jeweils durch die Hersteller zugesicherten Dosen geliefert werden. Die
zugesagten Lieferungen sind in Abbildung 2 dargestellt.

\textbf{Abb. 2: Bis Quartal 4 zugesagte Impfdosen, Quartal 1 ink. 2020}

\includegraphics{impfmodellierung_report_files/figure-latex/hersteller-1.pdf}

\hypertarget{modellierung}{%
\section{Modellierung}\label{modellierung}}

In unseren Modellierungen für die verschiedenen Leistungsfähigkeits- und
Lieferszenarien gehen wir zunächst von den Kapazitäten der Impfzentren
aus. Verfügbare monatliche Dosen werden im ersten Schritt für die
Auslastung der Impfzentren benutzt, bis diese Kapazität (z.B. 200.000
Impfungen täglich im Regelbetrieb) erschöpft ist. Die überzähligen Dosen
werden dann ärztlichen Praxen zugerechnet, die diese dezentral verimpfen
können. Wir berechnen, wie viele impfende Praxen für einen Monat nötig
sind, um die Impfzentren zu unterstützen und die vorhandenen
Impfstoffdosen auszuschöpfen.

In der aktuellen Darstellung wird ohne weitere Rückstellungen für die
Folgeimpfung gerechnet und bisher zurückgestellte Impfdosen auf die
nächsten 21 Tage verteilt verimpft.

\hypertarget{ergebnisse}{%
\section{Ergebnisse}\label{ergebnisse}}

\textbf{Abb. 4: Modellierte Auslastung der Impfzentren nach KW}

\includegraphics{impfmodellierung_report_files/figure-latex/outputplot1-1.pdf}

Es wird deutlich, dass bereits ab April die vertragsärztlichen Praxen
unbedingt einspringen müssen, um die gelieferten Impfstoffe zügig zu
verimpfen. Ab Juli bzw. im 3. Quartal sollten sämtliche geeignete Praxen
Impfungen durchführen.

\textbf{Abb. 5: Modellierte Unterstützungsbedarf durch Vertragsärzte
nach KW}

\includegraphics{impfmodellierung_report_files/figure-latex/outputplot2-1.pdf}

\end{document}
